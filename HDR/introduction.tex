La définition simple de l'hétérogénéité ("Manque d'unité, composé d'éléments de nature diverse", Dictionnaire Larousse) n'est pas suffisamment développée pour qualifier les différents défis liés à l'hétérogénéité dans les systèmes distribués. Pour cela, nous devons étendre et catégoriser les différents mécanismes liés à l'hétérogénéité.

Une première catégorie est issue directement de la définition simple ci-dessus, et représente les variations des équipements composant un système informatique. Sous cette optique, l'hétérogénéité se présente comme une conséquence de la différente construction des dispositifs qui composent un système, notamment leur composition matérielle (processeurs, mémoire) et leur capacité de calcul. Ainsi, un système distribué composé d'équipements identiques (caractérisé par l'absence d'hétérogénéité matérielle) peut supposer un traitement uniforme des tâches de calcul, traitement de données et donc simplifier la gestion des applications qu'y tournent. Au contraire, un système composé par des équipements hétérogènes, même partiellement, doit être conscient de cette différence et prévoir des mécanismes pour garantir l'exécution des applications : ordonnancement sensible aux capacités des équipements, synchronisation des tâches dépendantes, équilibrage et migration de charge, etc.  Même étant très réductrice, cette catégorie est souvent utilisée pour qualifier des équipements : machines parallèles (symétriques), clusters, grilles de calcul, réseaux P2P, etc.

À l'hétérogénéité matérielle s'ajoute le problème de l'hétérogénéité des communications, qui peut être causé autant par la diversité matérielle (par exemple, en utilisant différents types de réseaux) ou par la distance géographique (qui impacte les temps de communication). On retrouve l'hétérogénéité des communications surtout dans les systèmes distribués à grande échelle (grilles, réseaux P2P) où souvent le temps de communication est un facteur non négligeable. La prise en charge de l'hétérogénéité des communications se fait notamment par l'optimisation des dépendances : limitation des communications sur les grandes distances, ordonnancement basé sur la topologie du réseau, recouvrement des communications par de calculs pouvant être effectués en parallèle, etc.

La diversité matérielle et la diversité des communications (matérielle ou spatiale) constituent les gros facteurs liés à l'hétérogénéité à un moment donné. Toutefois, ce ne sont pas les seuls éléments impactant une application. En effet, nous devons assurer le fonctionnement d'un système distribué pendant toute la durée de l'exécution de l'application, et cette hétérogénéité temporelle est issu de la dynamicité d'opération d'un tel système. Plus exactement, les systèmes distribués peuvent être impactés par le départ ou l'arrivé de ressources, faisant ainsi la prise en compte de ces facteurs un besoin essentiel pour le bon fonctionnement d'un système et d'une application. La prise en charge de la dynamicité implique plusieurs éléments : la surveillance des ressources et la détection des pannes/états invalides, le suivi et la récupération des tâches attribuées à des éléments disparus et aussi le rééquilibrage de charge dans le cas d'une augmentation des ressources disponibles. En effet, la dynamicité temporelle affecte aussi l'utilisation des ressources, car même sans une défaillance un système doit pouvoir être amené à gérer plusieurs tâches indépendantes, chacune avec des besoins propres.


