% !TeX spellcheck = fr_FR

Tout au long de ce document j'ai essayé de mettre en évidence les différentes façons dont l'hétérogénéité peut affecter l'opération d'un système ou d'une application, ainsi que mes contributions pour leur prise en charge.

L'hétérogénéité peut se présenter sous différentes formes, demandant des actions spécifiques selon l'application, l'environnement et le contexte d'utilisation. Une première catégorie représente les \textit{variations matérielles} des composants d'un système informatique. À l'hétérogénéité matérielle s'ajoutent les problèmes de l'\textit{hétérogénéité des communications}, dont la prise en charge se fait notamment par l'optimisation des échanges des messages (Chapitre \ref{chap:grids}), et \textit{l'hétérogénéité des tâches}, résultat d'une distribution déséquilibrée de charge entre les différents ressources participant à un calcul (Chapitre \ref{chap:amide}). 

Il faut également être capable de supporter les variations des ressources tout au long de l'exécution, vu que \textit{l'hétérogénéité issue de la dynamique} (Chapitre \ref{chap:hadoop}) d'un système peut prendre différentes formes telles que le départ ou l'arrivé de ressources ou bien par leur changement d'état ou de capacité. Finalement, \textit{l'hétérogénéité des données} (Chapitre \ref{chap:grappes}) peut aussi impacter le développement d'une application à cause de leur variété et des spécificités d'accès selon les sources : des objets en mémoire, des fichiers, des URIs ou des requêtes distantes (RPC, Web services, etc.). 

Les travaux que j'ai conduit au fil des années touchent une ou plusieurs de ces manifestations de l'hétérogénéité, et dans ce document j'ai voulu montrer des cas représentatifs de ces travaux. Ainsi, au cours de la première partie de ce mémoire, j'ai présenté des algorithmes issus des mes travaux dont l'objectif était la compréhension des facteurs qui impactent les opérations de communication collective dans les \textit{grids}. J'ai pu exploiter ainsi des méthodes de mesure de performance et de découverte de la topologie réseau, et cela dans le but d'optimiser ces opérations mais aussi afin de pouvoir estimer leur performance avec une haute précision. Avec les années, mes travaux de recherche dans ce domaine se sont déportés vers l'analyse de performance des applications et systèmes, mais ayant toujours comme objectif l'optimisation des performances.

Dans la deuxième partie, j'ai utilisé comme exemple les efforts faits pour gérer l'hétérogénéité des tâches lors de la parallélisation et la gestion de l'exécution distribuée d'une application en biochimie. À partir d'une exécution monolithique, j'ai participé au développent de stratégies visant à découper l'espace de données et d'ainsi permettre la création de tâches de calcul pouvant être exécutées en parallèle. Ceci a été accompagné par le développement d'une plateforme de déploiement capable de gérer l'exécution des tâches distribuées sur  un \textit{cluster} HPC ou sur un ensemble de n{\oe}uds volontaires. Bien que relativement simple d'un point de vue technique, les activités développées pendant le co-encadrement de cette thèse m'ont apporté du recul vis-à-vis des processus nécessaires à la parallélisation et au déploiement d'une application tiers.

La troisième partie de ce mémoire se consacre à la dynamique des ressources et aux stratégies pour sa prise en charge. Ceci est illustré par une expérience visant à améliorer le comportement de la plateforme \textit{big data} Apache Hadoop dans les environnements hétérogènes et dynamiques (qu'on nomme ici environnements pervasifs). En s'aidant d'un mécanisme de collecte d'informations sur le contexte des ressources, il a été possible de modifier ce \textit{framework} et ainsi d'adapter la gestion des tâches aux ressources disponibles à chaque instant. Les concepts développés ici ont fortement influencé l'ensemble de mes activités de recherche, qui depuis se sont orientées vers le support à la dynamique des ressources dans les milieux hétérogènes. Ce n'est pas par hasard si mes recherches en ce moment se concentrent sur le \textit{fog computing}.    

La quatrième partie présente la spécification d'une base documentaire permettant l'accès transparent aux sources de données, peu importe leur nature (fichiers, flux, bases de données, etc.). Pour cela, j'ai présenté une spécification pour la construction d'un réseau hiérarchique pouvant héberger cette base documentaire. Les contributions de cette partie sont peut-être moins présentes dans ma recherche du fait d'avoir une préférence sur les réseaux P2P au détriment des réseaux hiérarchiques, toutefois la problématique liée à la diversité des sources de données reste un défi majeur que je fais ressortir notamment dans mes activités d'enseignement.

Finalement, dans la dernière partie de cette habilitation, j'ai présenté en détails la plateforme expérimentale de calcul distribué CloudFIT. Dans un premier moment j'ai montré l'architecture et les mécanismes de communication et de gestion des n{oe}uds, spécifications que jusqu'à présent n'avaient pas été publiées. Par la suite, j'ai introduit les stratégies pour le support à l'ordonnancement adapté au contexte qui font déjà partie de CloudFIT ou qui seront implémentées dans un avenir proche. Cette partie se termine par la présentation de deux exemples d'utilisation de CloudFIT, tous les deux issus de projets internationaux dont j'ai participé. Plus qu'un récapitulatif, ce chapitre a présenté l'évolution de  CloudFIT au fil des années, et son rôle en tant que est devenu une plateforme expérimentale pour mes recherches dans les domaines de l'IoT et du  \textit{fog computing} et l'Internet des Objets (\textit{Internet of Things} - IoT). .

L'ensemble de mes travaux a été à l'origine d'un nombre significatif de publications internationales (11 revues, 37 conférences, 3 posters, 3 chapitres de livre) et nationales (2 revues, 15 conférences, 3 posters). J'ai aussi participé au co-encadrements de deux thèses de doctorat et d'un post-doctorat : la thèse de Romain Vasseur (thèse CIFRE, dirigée par Manuel Dauchez et aussi co-encadrée par Stéphanie Baud), celle de Thierno Ahmadou Diallo (thèse en co-tutelle avec l'Université Cheikh Anta Diop - Sénégal, dirigée par Olivier Flauzac et Samba Ndiaye), et le stage post-doctoral de Iyad Alshabani (dans le cadre du projet ANR USS-SIMGRID). 

J'ai également été en charge de l'élaboration de trois projets de collaboration internationale entre l'Université de Reims et l'Amérique du Sud : le projet STIC-AmSud PER-MARE (2013-2014, avec l'Universidad de la República - Uruguay et l'Universidade Federal de Santa Maria - Brésil), le projet STIC-AmSud CC-SEM (2017-2018, avec L'Universidad de Buenos Aires - Argentine et l'Universidad de la República - Uruguay), puis le projet CAPES-Cofecub MESO (2017-2020, avec l'Université de la Réunion et  l'Universidade Federal de Santa Maria - Brésil). Dans tous ces projets j'ai exercé le rôle de coordinateur pour l'équipe de Reims et, dans le cas du projet PER-MARE, j'ai été aussi le coordinateur international.

Les perspectives associées à mes travaux sont nombreuses, d'une part à cause des projets en cours et d'autre part par les opportunités d'innovation dans les domaines de mes activités. Néanmoins, je souhaite à l'avenir privilégier trois thèmes.

\subsection*{Le \textit{fog computing} et les environnements pervasifs}

Les environnements pervasifs et le \textit{fog computing} constituent pour moi des sujets de recherche prioritaires. Tout d'abord, ils représentent l'opportunité de poursuivre les sujets de recherche auxquels je me suis dédié ces dernières années : la prise en compte de l'hétérogénéité des ressources, de l'hétérogénéité des tâches, de la volatilité, de l'ordonnancement sensible au contexte et du calcul multi-échelle. 

Le grand attractif des environnement pervasifs et \textit{fog computing} reste toutefois la diversité de problèmes ouverts. Le manque de standardisation n'a pas encore permis un consensus entre les chercheurs, résultant en multiples visions qui ne sont pas toujours compatibles. Même si une spécification voit le jour demain, elle sera forcément réductrice et donnera naissance à de nouveaux verrous scientifiques qu'il faudra explorer : une spécification trop orientée "services dans le \textit{edge}" laissera forcément du vide en ce qui concerne la migration de tâches et de micro-services ; des solutions réseaux contrôlées par les opérateurs auront par conséquence la multiplication des recherches autour des réseaux éphémères contrôlés par les applications. 
 
 À mon avis le principal défi à répondre est celui de la perméabilité des frontières entre l'Internet des objets, le \textit{fog computing} et le \textit{cloud}. Une différentiation trop marquée serait rapidement dépassée du fait de la multiplication des dispositifs IoT et de leur montée en performance. Une perméabilité trop importante induirait plus de défis liés à l'hétérogénéité que nécessaire. En fondant mes recherches dans ce thème, j'espère pouvoir apporter ma contribution.
 
Évidemment, les objectifs à court terme incluent ceux décrits dans les chapitres précédents : continuer le développement de la plateforme CloudFIT, poursuivre la recherche sur le support à la virtualisation et aux micro-services dans les nano-ordinateurs, etc. L'arrivée d'une doctorante en cotutelle et d'un post-doctorant à la fin 2017, dans le cadre du projet CAPES-Cofecub MESO, sera aussi l'occasion de tester l'exécution distribuée de modèles atmosphériques spécifiques pour la couche d'Ozone. Ces projets forment donc le socle de base pour mes recherches à plus long terme sur le \textit{fog computing} et les environnements pervasifs.  

\subsection*{L'Internet of Things et les \textit{smart cities}}

L'IoT est sans aucun doute un sujet prioritaire de ma recherche, autant parce que l'IoT représente un domaine où l'hétérogénéité se présente sous toutes ses différentes facettes, mais aussi parce que l'IoT est une cible prioritaire pour la compréhension du \textit{fog computing}. En effet, dans ma vision, le progrès de l'électronique embarquée fera qu'au moins une partie des dispositifs IoT évoluera jusqu'à devenir des véritables n{\oe}uds de calcul. À partir de ce moment, ces n{\oe}uds seront capables d'effectuer une partie des tâches qui aujourd'hui sont transmises au \textit{cloud} ou aux dispositifs à la périphérie des réseaux. Apprendre à contrôler et à orchestrer le calcul sur ces dispositifs marqués par une forte hétérogénéité serait une étape de plus dans la consolidation d'un véritable \textit{fog computing}.

Les recherches en IoT peuvent aussi s'associer à celles sur les \textit{smart cities}. Cette association nous ouvre un volet applicatif très intéressant, non seulement par la possibilité de mettre en pratique des algorithmes mais aussi à cause de leur impact sur notre vie quotidienne. Ce n'est pas par hasard que plusieurs projets que j'intègre sont orientés dans ce sens. Dans le cas du projet STIC-AmSud CC-SEM, l'objectif principal est le développement d’une plateforme intégrée pour la surveillance et le contrôle de la consommation électrique dans les milieux urbains. Ceci implique donc le développement des dispositifs électroniques pour la mesure de la consommation mais aussi la mise en place du réseau de collecte et d'analyse des données. 

On retrouve aussi la thématique des \textit{smart cities} sous le nom \textit{smart agriculture},  l'un des axes prioritaires de l'Université de Reims Champagne-Ardenne. Dans ce cas, je suis engagé dans un projet avec les chercheurs de l'Unité de Recherche Vignes et Vin de Champagne afin de mettre en place un réseau de capteurs déployé au pied des vignes. Ce réseau IoT permettra la surveillance de paramètres tels que l'humidité du sol, la luminosité et la température autour des plantes, données qui seront ensuite analysées, notamment afin de corréler ces données aux apparitions de certaines maladies. 

Une autre opportunité de recherche à court et moyen terme viendra avec le projet ANR autour de la \textit{Green IT} qui sera soumis cette année. Dans ce projet, nous nous intéressons au développement de stratégies d'ordonnancement basés sur le contexte et sur des objectifs de consommation préétablis afin d'aider à la réduction de l'empreinte énergétique des applications. Ces stratégies peuvent donc être appliquées au développement de services et d'applications mobiles, en utilisant par exemple la migration de composants et d'applications à des fins d'économie d'énergie. 

Les recherches sur l'\textit{Internet of Things} et sur les \textit{smart cities} représentent donc des sujets de recherche à forte valorisation à le court terme, avec l'avantage d'être toujours alignés avec mes objectifs principaux à long terme. 

\subsection*{Le calcul distribué et le big data}

Le dernier thème prioritaire de ma recherche concerne le calcul distribué et le \textit{big data}. Pour être plus précis, je considère que ces deux sujets sont de moins en moins dissociés, les avancées dans le domaine du \textit{big data} étant calquées essentiellement sur des plateformes et techniques issues du calcul distribué. De même, la recherche sur le calcul distribué ne peut plus ignorer les couts liés à la gestion et à la transmission des grandes masses de données, un paramètre qui tend à être encore plus décisif dans le cas des environnements hétérogènes tels que le \textit{fog computing}. 

Les outils pour le calcul distribué (dont le \textit{big data}) sont toutefois encore trop dépendants d'infrastructures stables et homogènes. Comme le prouve notre travail sur le framework Hadoop, il est nécessaire de repenser ou, tout au moins, d'adapter les plateformes et frameworks de manière à mieux gérer l'hétérogénéité et la dynamicité des ressources. En effet, je surveille de près d'autres plateformes telles que Apache Storm, vu que ces plateformes peuvent dans certains cas répondre plus efficacement à certaines exigences d'un déploiement \textit{fog computing} ou, au contraire, bénéficier des techniques développées pour CloudFIT.

Cette thématique présente aussi un volet pratique qui peut venir à supporter mes recherches futures. En effet, la 2\up{ème} phase du projet STIC-AmSud CC-SEM, dédié à la consommation électrique, doit s'élargir afin d'inclure des éléments de traitement \textit{big data} et d'apprentissage automatique (\textit{deep learning}). Le projet CAPES-Cofecub MESO, de son côté, dépend de l'analyse de données afin de comprendre les phénomènes atmosphériques et créer des modèles de prédiction fiables. Afin de remplir ces tâches, je compte autant sur l'expertise des membres de mon équipe sur le \textit{deep learning} que sur l'expérience obtenue pendant mes interventions dans le module "Outils Big Data", lequel j'enseigne aux étudiants en Master 2 Informatique et en Master 2 Statistique pour l'Évaluation et Prospective. 

 En perfectionnant mes connaissances sur les outils et les algorithmes de traitement des données, j'espère bien élargir les possibilités de mes recherches futures, tout en contribuant au développement des thématiques telles que le \textit{fog computing }qui est au centre de mes intérêts.
%
%
%\subsection*{Efficacité énergétique dans le cadre des \textit{Smart cities} et de l'informatique mobile}
%
%L'efficacité énergétique est une thématique forte ces dernières années, avec beaucoup de contribution autant du côté des équipements comme celui des réseaux de distribution mais aussi dans le domaine des \textit{smart cities}. En rejoignant le projet STIC-AmSud CC-SEM, j'ai pu constater que le domaine des \textit{smart cities} est à l'intersection de plusieurs disciplines avec qui j'ai souvent pu travailler : les systèmes distribués, l'IoT et les réseaux de capteurs, les algorithmes de récupération et de traitement de données, etc. De par ma participation à ce projet, les perspectives de recherche à court et moyen terme sont nombreuses.
%
%
%
%
%
%
%\subsection*{Expérimentation et développement de \textit{middlewares} pour le \textit{fog \\ computing}}
%
%Les problèmes de recherche autour du \textit{fog computing} et des environnements pervasifs en général sont multiples. 
%
%Le développement de la plateforme CloudFIT offre à la fois un outil pour la recherche et un cadre pour l'expérimentation de nouvelles techniques. Tout naturellement, je souhaite continuer son développement, intégrant des éléments issus de mes recherches autour du \textit{fog computing} mais aussi autour des projets que je participe. Comme indiqué dans le chapitre dédié à CloudFIT, cette plateforme offre plusieurs outils nécessaires pour la création d'une organisation multi-échelle du réseau, ce que permettrait une meilleure gestion des ressources. La mise en place de mécanismes pour la création à la volée de communautés CloudFIT fait partie de mes priorités immédiates, tout comme le prototypage de la technique de renforcement de la localité des données proposé dans le même chapitre.
%
%Plus à moyen terme, j'envisage rajouter des mécanismes pour la migration des tâches, ce qui permettrait l'expérimentation de stratégies spécifiques pour la distribution des tâches. Les verrous scientifiques dans ce domaine sont nombreux car la migration doit être précédée par une étude des application et des techniques pour garder leur cohésion malgré la migration entre les ressources. Ainsi, par exemple, on pourrait faire usage de machines virtuelles de type conteneur, le moyen le plus simple à mon avis pour effectuer la migration de tâches en exécution. D'autres pistes à suivre concerneraient l'usage de micro-services ou bien une combinaison entre ces deux techniques. Cette activité intégrera probablement les le cadre de la collaboration avec l'Université Paris 1 autour du déploiement de conteneurs sur des nano-ordinateurs. 
%
%
%
%Mon intérêt pour cette thématique ne se limite pas à CloudFIT. En effet, je surveille de près d'autres plateformes telles que Apache Storm, vu que ces plateformes peuvent dans certains cas répondre plus efficacement à certaines exigences d'un déploiement \textit{fog computing}. L'expérimentation et le développement sur ces plateformes permet donc d'obtenir des éléments de comparaison pour CloudFIT et de mieux avancer dans mes recherches dans ce domaine.
%
%
%
%\subsection*{Applications de l'\textit{Internet of Things} et des réseaux de capteurs}
%
%Au delà du développement de \textit{middlwares} pour le \textit{fog computing}, je souhaite continuer à développer des équipements et des outils pour l'IoT, notamment dans le cadre de l'axe "\textit{smart agriculture}" promu par l'Université de Reims Champagne-Ardenne. Ainsi, nous sommes, par exemple, en négociation avec les chercheurs de l'Unité de Recherche Vignes et Vin de Champagne afin de développer un réseau de capteurs (équipement, réseaux et outils) pouvant être déployé au pied des vignes, permettant ainsi de surveiller des paramètres tels que l'humidité du sol mais aussi la luminosité, la température et l'humidité de l'air autour des plantes. D'autre part, nous souhaitons aussi développer des techniques d'analyse de ces données afin de connaître le stress hydrique des plantes et de les corréler avec l'apparition de certaines maladies. 
%
%Je souhaite aussi compléter le développement et le déploiement des micro-contrôleurs Arduino équipés du capteur UV de la gamme ML8511, afin de les intégrer à un réseau de surveillance de l'Ozone Antarctique piloté par CloudFIT. Inséré dans le cadre du projet CAPES-Cofecub MESO, ce réseau sera utilisé en complément des mesures effectuées par des instruments spécifiques tels que les spectre-photomètres Dobson et Brewer. Après une première phase de calibration et tests de durabilité (ces capteurs seront déployés à l'air libre), nous espérons pouvoir disseminer  ces capteurs sur une large zone géographique et ainsi améliorer de manière peu onéreuse la couverture des équipements actuelles.
%
%
% 
%
%%TODO mettre ça dans les conclusions finales ? 
%
%%
%%
%%Ainsi, au cours des travaux antérieurs à ma thèse je m'étais penché sur la détection des pannes et leur impact sur les algorithmes de Consensus, ce qui m'a permis entre-autre de faire un séjour à l'EPFL à Lausanne pour travailler avec le professeur André Schiper. Bien que mes recherches se sont diversifiées au fil des années, c'est une thématique qui revient ponctuellement dans mes travaux, comme dans le cas de l'étude sur la diffusion avec ordre total (REFF).
%%
%%Pendant ma thèse les travaux se sont concentrés sur la modélisation des performances des communications à grande échelle, en profitant de l'essor des recherches sur le grid computing et le lancement du réseau Grid'5000. Cette thématique m'a permis de tisser d'importantes collaborations qui ont perduré après la thèse, comme l'attestent les travaux XXXX et YYY.
%%
%%L'arrivée à Reims marque un tournant dans mes recherches, du fait d'être intégré à une équipe spécialisée dans le calcul distribué et HPC. Tout en poursuivant une partie des travaux sur la modélisation des performances, j'ai été peu à peu  
%%
%%
%%
%%Bien sûr, ce travail n'a pas été le seul dans ce domaine. 
%%	