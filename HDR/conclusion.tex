% !TeX spellcheck = fr_FR

%TODO mettre ça dans les conclusions finales ? 
Ainsi, au cours des travaux antérieurs à ma thèse je m'étais penché sur la détection des pannes et leur impact sur les algorithmes de Consensus, ce qui m'a permis entre-autre de faire un séjour à l'EPFL à Lausanne pour travailler avec le professeur André Schiper. Bien que mes recherches se sont diversifiées au fil des années, c'est une thématique qui revient ponctuellement dans mes travaux, comme dans le cas de l'étude sur la diffusion avec ordre total (REFF).

Pendant ma thèse les travaux se sont concentrés sur la modélisation des performances des communications à grande échelle, en profitant de l'essor des recherches sur le grid computing et le lancement du réseau Grid'5000. Cette thématique m'a permis de tisser d'importantes collaborations qui ont perduré après la thèse, comme l'attestent les travaux XXXX et YYY.

L'arrivée à Reims marque un tournant dans mes recherches, du fait d'être intégré à une équipe spécialisée dans le calcul distribué et HPC. Tout en poursuivant une partie des travaux sur la modélisation des performances, j'ai été peu à peu  
